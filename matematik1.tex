\textless\documentclass[a4paper,10pt]{book}
\usepackage{ucs}
\usepackage[utf8x]{inputenc}
\usepackage{amsmath}
\usepackage{amsfonts}
\usepackage{amsthm}
\usepackage[lflt]{floatflt}
\usepackage{/usr/share/texmf/tex/latex/asymptote/asymptote}
\usepackage{pstricks,pst-node}
\usepackage{mathpazo}
\setcounter{chapter}{2}
\voffset	=-10.0mm
\hoffset	=-0.5in
\textheight	=614pt
\footskip	=-40mm
%\topmargin	=30mm	
%\oddsidemargin	=-10.mm		
\evensidemargin	=5.mm		
%\headheight	=0.mm
%\headsep	=0.mm
%\textheight	=220.mm
\textwidth	=180.mm
\begin{document}
\psset{nodesep=3pt}
\def\xstrut{\vphantom{\frac{(A)^1}{(B)^1}}}
\title{Ders Notları}
\date{}
\def\chaptername{Bölüm}
\newtheoremstyle{italik}{}{}{\normalfont}{}{\bfseries}{.}{ }{} % basic definitions, use roman font: 
\theoremstyle{italik}
\newtheorem{ornek}{Örnek}[chapter]
\newtheorem*{cozum}{Çözüm}
\newtheorem*{tanim}{Tanım}[chapter]
\maketitle
\chapter{Sayılar}
\chapter{Say}

\section{Sayılar}
\subsection{Doğal Sayılar}
$\mathbb{N}=\left\{0,1,2,3,4,5,6,7,8,9,10,11,12,\ldots\right\}$ kümesine "Doğal Sayılar Kümesi" denir ve $\mathbb{N}$ ile
gösterilir.
\subsubsection{Doğal Sayıların Özellikleri}
$\forall a,b,c \in \mathbb{N}$ için
\begin{enumerate}
	\item $a=a$ (yansıma özelliği)
	\item $a=b \Leftrightarrow b=a$ dır. (simetri özelliği)
	\item $a=b$ ve $b=c \Rightarrow a=c$ dir. (geçişme özelliği)
	\item $a=b \Leftrightarrow a+c=b+c$ dir.
	\item $a=b \Leftrightarrow a \cdot c=b \cdot c$ dir.
\end{enumerate}
özellikleri sağlanır.

\subsubsection{Sayma Sayıları Kümesi}
$\mathbb{N}^+=\left\{1,2,3,4,5,6,7,8,9,10,11,\ldots\right\}$ kümesine "Sayma Sayıları Kümesi" denir.Dikkat edilirse,
Sayma sayıları kümesi Doğal sayılar kümesinden $0$ sayısının çıkarılmasıyla elde edilmiştir.

\subsubsection{Çift Doğal Sayılar}
$ 2 $ ile bölündüğünde $ 0 $ kalanını veren doğal sayılara "Çift Doğal Sayılar" denir.Tanımdan da anlaşılacağı gibi
Çift doğal sayılar kümesi $ \mathbb{N}_{\text{ç}} $ ile gösterilirse,
$$ \mathbb{N}_{\text{ç}}=\left\{0,2,4,6,8,10,12,14,\ldots\right\} $$
dir.
\subsubsection{Tek Doğal Sayılar}
$ 2 $ ile bölündüğünde $ 1 $ kalanını veren doğal sayılara "Tek Doğal Sayılar" denir.Tek doğal sayılar kümesi 
$ \mathbb{N}_{t} $ ile gösterilirse,
$$ \mathbb{N}_{t}=\left\{1,3,5,7,9,11,\ldots\right\} $$
dir.

\subsubsection{Asal Sayılar}
\begin{tanim}
	\quad\\
	Pozitif bölenler kümesi iki elemanlı olan doğal sayılara "Asal Sayılar" denir.
\end{tanim}

\begin{ornek}
	\quad\\
	$ 1 $ sayısını pozitif bölenleri kümesi $A$ ile gösterilirse,
	$$ A=\left\{1\right\} $$
	dır.
	Bu küme bir elemanlı olduğundan $ 1 $ sayısı asal sayı \underline{değildir}.\\
	Benzer şekilde $ 2 $ sayısının pozitif bölenleri kümesi $ B $ ile gösterilirse,
	$$ B=\left\{1,2\right\} $$
	olur ki bu $ 2 $ sayısının asal olduğunu gösterir.\\
	$ 9 $ sayısının pozitif bölenleri kümesi $ C $ ile gösterilirse,
	$$ B=\left\{1,3,9\right\} $$
	olur.$ 3 $ sayısının asal sayı değildir.
\end{ornek}
Bu bilgiler ışığında, Asal sayılar kümesi $ A $ ile gösterilirse,
$$ A=\left\{2,3,5,7,11,13,17,19,\ldots\right\} $$
dır.
Asal sayılar, 1'den ve kendisinden başka hiç bir pozitif böleni olmayan sayılardır diye tanımlanırsa hata yapılmış
olur.Gerçekten $ 1 $ sayısı hem $ 1 $'e hem de kendine bölünür ve de başka pozitif böleni yoktur.Fakat asal sayıların
tanımı gereği $ 1 $'in asal sayı olmadığını daha önce görmüştük.
Buna rağmen tanım aşağıdaki haliyle doğrudur.
\begin{tanim}
	\quad\\
	$ 1 $'den ve kendisinden başka hiçbir pozitif sayıya bölünmeyen,$ 1 $'den büyük doğal sayılara \textbf{Asal Sayılar}
	denir.
\end{tanim}
\subsubsection{Aralarında Asal Sayılar}
$ 1 $'den başka pozitif ortak böleni olmayan iki doğal sayıya \textbf{Aralarında asaldırlar} denir.
\begin{ornek}
	3 ile 5 sayıları arlarında asaldırlar.\\
	4 ile 5 sayıları aralarında asaldırlar.\\
	9 ile 4 sayıları aralarında asaldırlar.
\end{ornek}
Örneklerden de anlaşılacağı gibi iki sayının aralarında asal sayılar olması için asal olmaları gerekmez.Gerçekten
4 ve 9 asal değillerdir ama 4 ile 9 aralarında asal sayılardır.
\subsection{Tam Sayılar}
$\mathbb{Z}=\left\{\ldots ,-7,-6,-5,-4,-3,-2,-1,0,1,2,3,4,5,6,7,\ldots\right\}$ kümesine "Tam Sayılar Kümesi" denir ve $\mathbb{Z}$ ile
gösterilir.
\chapter{Cebir}
Cebir,bilinmeyen değerlerin,işaretler ve harflerle sembolize edilmesi esasına dayanır \\[1.5cm]
\newrgbcolor{lila}{0.6 0.2 0.5}
\newrgbcolor{darkyellow}{1 0.9 0}
\def\xstrut{\vphantom{\frac{(A)^1}{(B)^1}}}
\begin{equation*}
	\rput[l](-2,2){\rnode{a}{Gerilim}}
	\rput[l](2,2){\rnode{c}{Direnç}}
	\rput[l](1,-2){\rnode{b}{Akım}}
	\rnode[t]{v}{\psframebox*[fillcolor=yellow,linestyle=none]{\xstrut V}} =
	\rnode[t]{i}{\psframebox*[fillcolor=cyan,linestyle=none]{\xstrut I}} \cdot
	\rnode[t]{r}{\psframebox*[fillcolor=green,linestyle=none]{\xstrut R}}
	\ncline[nodesep=3pt]{<->}{a}{v}
	\ncline[nodesep=3pt]{<->}{b}{i}
	\ncline[nodesep=3pt]{<->}{c}{r}
\end{equation*}\\[0.25cm]
Yukarıda ki örnekten de anlaşılabileceği gibi $V,I$ ve $R$ sembolleri sayıları temsil etmektedir.Örneğin $V$ gerilimi
$I$ devreden geçen akımı $R$ ise direnci gösterir.
\end{document}
