\documentclass[a4paper,10pt]{book}
\usepackage{ucs}
\usepackage[utf8x]{inputenc}
\usepackage{amsmath}
\usepackage{amsfonts}
\usepackage{amsthm}
\usepackage[lflt]{floatflt}
\usepackage{/usr/share/texmf/tex/latex/asymptote/asymptote}
\setcounter{chapter}{2}
\voffset	=-10.0mm
\hoffset	=-0.5in
\textheight	=614pt
\footskip	=-40mm
%\topmargin	=30mm	
%\oddsidemargin	=-10.mm		
\evensidemargin	=5.mm		
%\headheight	=0.mm
%\headsep	=0.mm
%\textheight	=220.mm
\textwidth	=180.mm
\title{Ders Notları}
\date{}
\def\chaptername{Bölüm}
\newtheoremstyle{italik}{}{}{\normalfont}{}{\bfseries}{.}{ }{} % basic definitions, use roman font: 
\theoremstyle{italik}
\newtheorem{ornek}{Örnek}[chapter]
\newtheorem*{cozum}{Çözüm}
\newtheorem*{tanim}{Tanım}
\begin{document}
\maketitle
\section{Doğrusal Denklem Sistemleri}
\subsection{Bir Bilinmeyenli Doğrusal Denklem}
$a,b\in\mathbb{R}$, $a\not=0$ ve $x$ bilinmeyen olmak üzere $ax+b=0$ şeklindeki denklemlere "bir bilinmeyenli
doğrusal denklem" denir.Bu denklemi sağlayan tek bir reel sayı vardır.

\begin{ornek}
\quad\\
$3x-2=10$ denkleminin çözüm kümesini bulunuz.
\end{ornek}

\begin{cozum}
	\begin{eqnarray*}
		3x-2         & = & 10 \\
		3x-2+2       & = & 10+2 \\
		3x           & = & 12 \\
		\frac{3x}{3} & = & \frac{12}{3} \\
		x            & = & 4
	\end{eqnarray*}
	o halde çözüm kümesi $$\text{Ç}=\left\{4\right\}$$ dir.
\end{cozum}

\begin{ornek}
	\quad\\
	$5-\left[3x-2(x-3)\right]=-x-1$ denkleminini çözüm kümesini bulunuz.
\end{ornek}

\begin{cozum}
	\begin{eqnarray*}
		5-\left[3x-2(x-3)\right] & = & -x-1 \\
		5-\left(3x-2x+6\right)   & = & -x-1 \\
		5-\left(x+6\right)       & = & -x-1 \\
		5-x-6                    & = & -x-1 \\
		-x-1                     & = & -x-1 \\
		0x                       & = & 0
	\end{eqnarray*}
	Burada denklemin çözüm kümesini bulmak için "hangi sayıları sıfırla çarparsam sonuç sıfır çıkar"
	sorusunu sorarız ve bu sorunun cevabı bize çözüm kümesinin
	$$\text{Ç}=\mathbb{R}$$
	olduğunu gösterir.
\end{cozum}

\begin{ornek}
	\quad\\
	$\frac{1}{2}x-2\left(x-3\right)=-\frac{3}{2}\left(x-6\right)$ denkleminini çözüm kümesini bulunuz.
\end{ornek}
\begin{cozum}
	\begin{eqnarray*}
		\frac{1}{2}x-2\left(x-3\right) &=& -\frac{3}{2}\left(x-6\right) \\
		\frac{1}{2}x-2x+6              &=& -\frac{3}{2}x+9 \\
		-\frac{3}{2}x+6                &=& -\frac{3}{2}x+9 \\
		0x                             &=& 3
	\end{eqnarray*}
	Bu denklemi sağlayan herhangi bir $x$ reel sayısı yoktur.O halde çözüm kümesi $$\text{Ç}=\phi$$ dir.
\end{cozum}

\subsection{İki Bilinmeyenli Doğrusal Denklem Sistemleri}
\begin{tanim}
	\quad\\
	$x$ ve $y$ bilinmeyenler $a,b,c,d\not=0$ olmak üzere
	\begin{eqnarray*}
		ax+by &=& e \\
		cx+dy &=& f
	\end{eqnarray*}
	sistemine \textsl{\textbf{iki bilinmeyenli doğrusal denklem sistemi}} denir.\\
\end{tanim}


	$
	\begin{aligned}
		2x+3y=6 \\
		3x+5y=7
	\end{aligned}
	$ \quad sistemi iki bilinmeyenli doğrusal denklem sistemidir.


\subsubsection{İki Bilinmeyenli Doğrusal Denklem Sisteminin Çözümü}
\textbf{1)\ Yoketme Yöntemi}
\renewcommand{\theenumi}{\roman{enumi}}
\begin{enumerate}
		\item					
			Denklemler,taraf tarafa toplanıldığında bilinmeyenlerden bir tanesi yok olacak şekilde
			uygun sayılarla çarpılır.
		\item
			Denklemler taraf tarafa toplanır.
		\item
			Toplama sonucunda elde edilen bir bilinmeyenli denklem çözülerek bilinmeyenlerden bir tanesi
			bulunur.
		\item
			Bu bilinmeyenin değeri denklemlerden herhangi birinde yerine koyularak diğer bilinmeyen
			bulunur.
\end{enumerate}



\begin{ornek}
	\quad\\
	$
	\begin{aligned}
		2x+5y=8 \\
		3x-2y=7
	\end{aligned}
	$
	\quad denklem sisteminin çözüm kümesini bulunuz.
\end{ornek}

\begin{cozum}
	$$
	\begin{array}{rclc}
		2/ \ 2x+5y &=& 8 & \\
		+\quad\  5/ \ 3x-2y &=& -7 &\\ \cline{1-4} 
	\end{array}
	$$
	çarpma işlemleri sonucunda
	$$
	\begin{array}{rclc}
		4x+10y &=& 16 & \\
		+\quad\  \ 15x-10y &=& -35 &\\ \cline{1-4}
		19x &=& -19 & \\
		x   &=& -1  &
	\end{array}
	$$
	elde edilir.$x$ değeri denklemlerden herhangi birinde yerine yazılarak,
	$$
	\begin{aligned}
		2x+5y=8 \Rightarrow 2\cdot(-1)+5y & = 8  \\
		                            -2+5y & = 8  \\
					       5y & = 10 \\
					        y & = 2  \\
	\end{aligned}
	$$
	$y$ bilinmeyeni bulunur ve denklemin çözüm kümesi
	$$
	\text{Ç}=\left\{(-1,2)\right\}
	$$
	dir.Bu sonuç $2x+5y=8$ ve $3x-2y=-7$ doğrularının $A(-1,2)$ noktasında kesiştiğini gösterir.
\end{cozum}
			

\begin{ornek}
	\quad\\
	$
	\begin{aligned}
		2x-3y  & = 6 \\
		-4x+6y & = 5
	\end{aligned}
	$
	\quad denklem sisteminin çözüm kümesini bulunuz.
\end{ornek}

\begin{cozum}
	$$
	\begin{array}{rclc}
		2/ \ 2x-3y          & = & 6  & \\
		+\quad\ \ -4x+6y    & = & 5  & \\ \cline{1-4}
		0x+0y		    & = & 17 & \\
		0		    & = & 17 &
	\end{array}
	$$
	Bu sonuç bize denklemleri sağlayan herhangi bir $(x,y)$ ikilisinin olmadığını yani bu denklemlere karşılık
	gelen doğruların kesişmediğini dolayısı ile paralel olduğunu gösterir.O halde
	$$
	\text{Ç}=\phi
	$$
	dir.
\end{cozum}

\textbf{2)\ Yerine Koyma Yöntemi}
\begin{enumerate}
	\item
		Denklemlerden biri kulanılarak herhengi bir bilinmeyenin değeri diğer bilinmeyen cinsinden elde
		edilir.
	\item
		Bulunan bu değer diğer denklemde yerine yazılarak bir bilinmeyenli doğrusal bir denklem elde edilir.
		Bu denklem çözülerek bilinmeyenlerden bir tanesi bulunur.
	\item
		Bulunan bu değer herhangi bir denklemde yerine yazılarak diğer bilinmeyenin değeri bulunur.
\end{enumerate}

\begin{ornek}
	\quad \\
	$
	\begin{aligned}
		-2x+y= & 5 \\
		x-3y=  & 4 \\
	\end{aligned}
	$
	doğrusal denkleminin çözüm kümesini bulunuz.
\end{ornek}

\begin{cozum}
	$$
	\begin{aligned}
		-2x+y=5 \Rightarrow y = & 5+2x
	\end{aligned}
	$$
	bulunan y değeri diğer denklemde yerine yazılarak,
	$$
	\begin{aligned}
		x-3y=4 \Rightarrow x-3\cdot \left(5+2x\right)= & 4  \\
						      x-15-6x= & 4  \\
						          -5x= & 19 \\
							    x= & -\frac{19}{5}
        \end{aligned}
	$$
	Denklemlerden herhangi bir tanesinde $x$ değeri yerine yazılırak $y$ değeri bulunur.Fakat $y$`nin 
	katsayısının $1$ olması sebebiyle $-2x+y=5$ denklemini kullanmak bize kolaylık sağlayacaktır.
	$$
	\begin{aligned}
		x=-\frac{19}{5} \Rightarrow -2\cdot -\frac{19}{5}+y= & 5 \\
						     \frac{38}{5}+y= & 5 \\
						                  y= & 5 - \frac{38}{5} \\
								  y= & \frac{25}{5} - \frac{38}{5} \\
								  y= & -\frac{-13}{5}
	\end{aligned}
	$$
	O halde çözüm kümesi,
	$$
	\text{Ç}=\left\{(-\frac{19}{5},-\frac{13}{5})\right\}
	$$
	dir.
\end{cozum}
       
	
				    
	
		
\subsection{Üç Bilinmeyenli Denklem Sistemleri}
$x,y,z$ bilinmeyenler olmak üzere 
\begin{eqnarray*}
	a_1x+b_1y+c_1z & = & d_1 \\
	a_2x+b_2y+c_2z & = & d_2 \\
	a_3x+b_3y+c_3z & = & d_3
\end{eqnarray*}
biçimindeki bir sisteme \textsl{\textbf{Üç Bilinmeyenli Doğrusal Denklem Sistemi}} denir.
\subsection{Uc Bilinmeyenli Doğrusal Denklem Sistemlerinin Cözümü}
Üc bilinmeyenli doğrusal denklem sistemlerinin cözümü için en genel yol denklemleri ikişerli gruplara ayırmak ve bu iki grupta aynı bilinmeyenleri yok edip iki bilinmeyenli doğrusal denklem sistemi elde etmektir.
\begin{ornek}
\end{ornek}
\begin{eqnarray*}
	 x+y+z & = & 3 \\
	 3x-2y+z & = & 1 \\
	 2x+y-3z & = & 0
\end{eqnarray*}
Üç bilinmeyenli doğrusal denklem sisteminin çözüm kümesini bulunuz.
\begin{equation*}
	\left.
	\begin{aligned}
		x+y+z & = & 3 \\
		3x-2y+z & = & 2
	\end{aligned}
	\right] \quad \text{I.Grup} \quad
	\left.
	\begin{aligned}
		3x-2y+z & = & 2 \\
		2x+y-3z & = & 0
	\end{aligned}
	\right] \quad \text{II. Grup}
\end{equation*}
I.Gruptan
\begin{alignat}{4}
	-1/ \  x+y+z= & 3 \nonumber \\
	\underline{+ \hspace{10mm} \strut 3x-2y+z= } & \underline{\strut 2 \qquad}\nonumber \\
	-2x-3y= & -1 
\end{alignat}
II.Gruptan
\begin{alignat}{4}
	3/ \  3x-2y+z= & 2 \nonumber \\
	\underline{+ \hspace{10mm} \strut 2x+y-3z= } & \underline{\strut 0 \qquad}\nonumber \\
	11x-5y= & 6 
\end{alignat}
(2.1) ve (2.2) nolu denklemler birleştirilerek
\begin{alignat}{4}
	-5/ \  -2x-3y= & -1 \nonumber \\
	\underline{+ \hspace{7mm} \strut -3/ \ 11x-5y= } & \underline{\strut 6 \qquad }\nonumber \\
	-23x= & -23 \nonumber \\
	x= & 1 
\end{alignat}
bulunur.Bulunan bu değer iki bilinmeyenli denklemlerden birinde yerine yazılarak,
\begin{align}
	2x-3y=-1 \Rightarrow 2.1-3y & =-1 \nonumber \\
	2-3y & =-1 \nonumber \\
	-3y & = -3  \nonumber \\
	y & = 1
\end{align}
Orijinal denklemlerin herhangi bir tanesine (2.3) ve (2.4) eşitlikleri yazılarak,
\begin{align}
	x+y+z=3 \Rightarrow 1+1+z & = 3 \nonumber \\
	2+z & =3 \nonumber \\
	z & =1  \nonumber  
\end{align}
bulunur.O halde çözüm kümesi
$$
\text{\c{C}}=\left\{(1,1,1)\right\}
$$ 
dir.
\begin{ornek}
\end{ornek}
\begin{eqnarray*}
	-x+y+z & = & 3 \\
	3x-2y-3z & = & -7 \\
	x+5y-4z & = & 9
\end{eqnarray*}
Üç bilinmeyenli doğrusal denklem sisteminin çözüm kümesini bulunuz.
\begin{equation*}
	\left.
	\begin{aligned}
		-x+y+z & = & 3 \\
		3x-2y-3z & = & -7
	\end{aligned}
	\right] \quad \text{I.Grup} \quad
	\left.
	\begin{aligned}
		3x-2y-3z & = & -7 \\
		x+5y-4z & = & 9
	\end{aligned}
	\right] \quad \text{II. Grup}
\end{equation*}
I.Gruptan
\begin{alignat}{4}
	3/ \  -x+y+z= & 3 \nonumber \\
	\underline{+ \hspace{10mm} \strut 3x-2y-3z= } & \underline{\strut -7 \qquad}\nonumber \\
	y= & 2
\end{alignat}
II.Gruptan
\begin{alignat}{4}
	3x-2y-3z= & -7 \nonumber \\
	\underline{+ \hspace{10mm}-3\ \ \strut x+5y-4z= } & \underline{\strut 9 \qquad}\nonumber \\
	-17y+9z= & -34 
\end{alignat}
(2.1) ve (2.2) nolu denklemler birleştirilerek,
\begin{align}
	-17y+9z=-34 \Rightarrow -17 \cdot 2+9z & =-34 \nonumber \\
	-34+9z & = -34 \nonumber \\
	9z & = 0  \nonumber \\ 
	z & = 0
\end{align}
Orijinal denklemlerin herhangi bir tanesine (2.3) ve (2.4) eşitlikleri yazılarak,
\begin{align}
	-x+y+z=3 \Rightarrow -x+2+0 & = 3 \nonumber \\
	-x & =1 \nonumber \\
	x & =-1  \nonumber  
\end{align}
bulunur.O halde çözüm kümesi
$$
\text{\c{C}}=\left\{(-1,2,0)\right\}
$$ 
dir.
\chapter{Matrisler ve Determinantlar}
\section{Matris Kavramı}
Bir tuhafiyecide satılan ürünleri listelemek istediğimizi düşünelim.Bunu yapmanın en iyi yöntemi
$$
\begin{array}{lcr}
	\quad & Mavi & Kırmızı \\
	\begin{array}{l} 
		Pantalon \\
		Etek 
	\end{array} & \left[
				\begin{array}{r}
					4 \\
					0
				\end{array}
		      \right. & \left.\begin{array}{l} 4 \\ 0 \end{array}\right]
\end{array}
$$
gibi bir yapı kullanmaktır.İşte bu şekildeki dikdörtgensel tablolara Matris adı verilir.
$$
\left[
\begin{array}{cccc}
	a_{11} & a_{12} & \ldots & a_{1n} \\
	a_{21} & a_{22} & \ldots & a_{2n} \\
	a_{i1} & a_{i2} & \vdots & a_{in} \\
	a_{m1} & a_{m2} & \ldots & a_{mn} 
\end{array}\right]_{mxn}
$$
biçimindeki dikdörtgensel tabloya m\ x n türünden bir matris denir ve kısaca,
$$
A=\left[a_{ij}\right]_{mxn}
$$
şeklinde gösterilir.
$ a_{ij} $ elemanında i ye satır indisi , j ye de  sütun indisi denir.$ a_{ij} $,matrisin i. satır, j. sütundaki elemanını göstermektedir.
\begin{ornek}
	$$
	A=\left[
	\begin{array}{cccc}
		3 & 5 & 2 \\
		-1 & 7 & 9 \\
		6 & 0 & 4 
	\end{array}
	\right]_{3\ x\ 3}
	$$
	matrisi verilsin.Burada,
	\begin{align*}
		a_{11}&=3 & a_{32}&=0 \\
		a_{23}&=9 & a_{13}&=2 \\
		a_{31}&=6 & & & \\
	\end{align*}
	dir.
\end{ornek}
\begin{tanim}
	\quad \\
	Bütün elemanları sıfır olan matrisi \textsl{\textbf{Sıfır Matris}} denir ve $0$ ile gösterilir.
\end{tanim}
$
A=\left[
\begin{array}{ccc}
	0 & 0 & 0 \\
	0 & 0 & 0 \\
\end{array}
\right]_{2\ x\ 3}
$
matrisi $2x3$ tipinde bir sıfır matrisdir.

\begin{tanim}
	\quad \\
	Bir matrisin satır sayısı, sütun sayısına eşitse bu matrise \textsl{\textbf{Karesel Matris}} denir.
\end{tanim}
$
A=\left[
\begin{array}{cc}
	5 & 1 \\
	2 & 4 \\
\end{array}
\right]_{2\ x\ 2}
$
matrisi karesel matrisdir.
\section{Bir Matrisin Transpozu}
Bir $ A=\left[a_{ij}\right]_{mxn} $ matrisinde her i için i. satırın i. sütun yapılması ile elde edilen matrise A matrisinin \textbf{transpozu} denir ve $ A^T $ ile gösterilir.
\begin{ornek}
\end{ornek}
$$
A=\left[\begin{array}{cccc}
	1 & 4 & 7 \\
	2 & 5 & 8 \\
	3 & 6 & 9 
\end{array}\right]_{3\ x\ 3}
\Rightarrow
A^{T}=\left[\begin{array}{cccc}
	1 & 2 & 3 \\
	4 & 5 & 6 \\
	7 & 8 & 9 
\end{array}\right]_{3\ x\ 3}
$$
dir.
\section{Determinantlar}
Determinant her bir kare matrisi bir reel sayıyya götüren bir fonksiyondur ve A herhangi bir kare matris olmak üzere
$$
det\left[A\right] \text{veya } |A|
$$ 
şeklinde gösterilir.
\subsection{2 x 2 Tipindeki Kare Matrislerin Determinantı}
$
A=\left[\begin{array}{cccc}
	a & b  \\
	c & d \\
\end{array}\right]_{2\ x\ 2}
$
bir 2 x 2 tipinde kare matris olmak üzere,
$$
det\left[A\right]=a\cdot d - b\cdot c
$$
dir.
\begin{ornek}
\end{ornek}
$
\begin{array}{ll}
	A=\left[\begin{array}{cccc}
		-3 & 4  \\
		-2 & 5 \\
	\end{array}\right]_{2\ x\ 2} \Rightarrow
	det\left[A\right]&=(-3)\cdot 5 - 4\cdot (-2) \\
	&=-7 \qquad \text{dir}.
\end{array}
$
\begin{ornek}
\end{ornek}
$
\begin{array}{ll}
	A=\left[\begin{array}{cccc}
		-10 & 5  \\
		-2 & 5 
	\end{array}\right]_{2\ x\ 2} \Rightarrow
	det\left[A\right]&=(-10)\cdot 5 - 5\cdot (-2) \\
	&=-40 \qquad \text{dir}.
\end{array}
$
\begin{ornek}
\end{ornek}
$
A=\left[\begin{array}{cccc}
	1 & x  \\
	2 & x^2 
\end{array}\right]_{2\ x\ 2}=3
$
ise $ x $ kaçtır.
\begin{align*}
	det[A]=1 \cdot x^2 - 2 \cdot x &= 3 \\
	x^2-2x-3 &=0 \\
	(x-3)(x+1)&=0 \quad \Rightarrow x=3\ \vee\ x=-1 \quad \text{dir.}
\end{align*}
\subsection{3 x 3 Tipindeki Kare Matrislerin Determinantı}
\subsubsection{Sarrus Kuralı}
Sarrus Kuralı 3 x 3 türündeki karesel matrislerin determinantinin hesaplanması için pratik yöntemdir.
Bu yöntemde matrisin ilk iki satırı(sütunu) matrisin altına(sağına) yazılır.Daha sonra (+) işaretli köşegen
üzerindeki elemanlar çarpılıp toplanmasıyla elde edilen sonuçtan, (-) işaretli köşegen üzerindeki elemanların çarpılıp
toplanmasıyla elde edilen sonuç çıkartılır.Bulunan değer matrisin determinantıdır.
Bir A matrisinin determinantı 
$$
det[A] \text{veya} |A|
$$
şeklinde gösterilir.\\
$
\begin{aligned}
	\begin{asy}
		size(4cm,0);
		pair z1=(1,1);
		pair z2=(1,0);
		pair z3=(1,-1);
		pair z4=(2,1);
		pair z5=(2.0);
		pair z6=(2,-1);
		pair z7=(3,1);
		pair z8=(3,0);
		pair z9=(3,-1);
		pair z10=(1,-2);
		pair z11=(1,-3);
		pair z12=(2,-2);
		pair z13=(2,-3);
		pair z14=(3,-2);
		pair z15=(3,-3);
		label("$a_{11}$",z1);
		label("$a_{21}$",z2);
		label("$a_{31}$",z3);
		label("$a_{12}$",z4);
		label("$a_{22}$",z5);
		label("$a_{32}$",z6);
		label("$a_{13}$",z7);
		label("$a_{23}$",z8);
		label("$a_{33}$",z9);
		label("$a_{11}$",z10);
		label("$a_{21}$",z11);
		label("$a_{12}$",z12);
		label("$a_{22}$",z13);
		label("$a_{13}$",z14);
		label("$a_{23}$",z15);
		draw((0.8,-1.5)--(3.2,-1.5));
		draw((0.5,1.3)--(0.5,-1.3));
		draw((3.5,1.3)--(3.5,-1.3));
		draw(z9--(0,1.8),Arrow,PenMargins);
		draw(z14--(0,0.8),Arrow,PenMargins);
		draw(z15--(0,-0.2),Arrow,PenMargins);
		draw(z7--(0,-1.7),red,Arrow,PenMargins);
		draw(z8--(0,-2.7),red,Arrow,PenMargins);
		draw(z9--(0,-3.7),red,Arrow,PenMargins);
		label("$+$",(-0.5,1.8));
		label("$+$",(-0.5,0.8));
		label("$+$",(-0.5,-0.2));
		label("$-$",(-0.5,-1.7));
		label("$-$",(-0.5,-2.7));
		label("$-$",(-0.5,-3.7));
	\end{asy}
\end{aligned}
=\left(a_{11} a_{22} a_{33}
+a_{21} a_{32} a_{13}
+a_{31} a_{12} a_{23} \right)
-
\left(a_{13} a_{22} a_{31}
+a_{23} a_{32} a_{11}
+a_{33} a_{12} a_{21} \right)
$
\begin{ornek}
	\quad \vspace{3mm} \\
	$
	A=\left[
	\begin{array}{lcr}
		-1           & \phantom{-}2  & \phantom{-}3 \\
		\phantom{-}2 & \phantom{-}5  &           -1 \\
		\phantom{-}3 & \phantom{-}1  & \phantom{-}2
	\end{array}\right]_{3\ x\ 3}
	$
	matrisinin determinantını hesaplayınız.
\end{ornek}

\begin{cozum}
	\quad \\
	$
	det[A]=
	\begin{aligned}
		\begin{asy}
			size(3cm,0);
			pair z1=(1,1);
			pair z2=(1,0);
			pair z3=(1,-1);
			pair z4=(2,1);
			pair z5=(2.0);
			pair z6=(2,-1);
			pair z7=(3,1);
			pair z8=(3,0);
			pair z9=(3,-1);
			pair z10=(1,-2);
			pair z11=(1,-3);
			pair z12=(2,-2);
			pair z13=(2,-3);
			pair z14=(3,-2);
			pair z15=(3,-3);
			label("$-1$",z1);
			label("$2$",z2);
			label("$3$",z3);
			label("$2$",z4);
			label("$5$",z5);
			label("$1$",z6);
			label("$3$",z7);
			label("$-1$",z8);
			label("$2$",z9);
			label("$-1$",z10);
			label("$2$",z11);
			label("$2$",z12);
			label("$5$",z13);
			label("$3$",z14);
			label("$-1$",z15);
			draw((0.8,-1.5)--(3.2,-1.5));
			draw((0.5,1.3)--(0.5,-1.3));
			draw((3.5,1.3)--(3.5,-1.3));
			draw(z9--(0,1.8),Arrow,PenMargins);
			draw(z14--(0,0.8),Arrow,PenMargins);
			draw(z15--(0,-0.2),Arrow,PenMargins);
			draw(z7--(0,-1.7),red,Arrow,PenMargins);
			draw(z8--(0,-2.7),red,Arrow,PenMargins);
			draw(z9--(0,-3.7),red,Arrow,PenMargins);
			label("$+$",(-0.5,1.8));
			label("$+$",(-0.5,0.8));
			label("$+$",(-0.5,-0.2));
			label("$-$",(-0.5,-1.7));
			label("$-$",(-0.5,-2.7));
			label("$-$",(-0.5,-3.7));
		\end{asy}
	\end{aligned}
	=\left[(-1) \cdot 5 \cdot 2 + 2 \cdot 1 \cdot 3 + 3 \cdot 2 \cdot (-1) \right]
	-\left[3 \cdot 5 \cdot 3 + (-1) \cdot 1 \cdot (-1) + 2 \cdot 2 \cdot 2 \right] \\
	\phantom{6} \hspace{43mm} = \left[-10+6+(-6)\right]-\left[45+1+8\right] \\
	\phantom{6} \hspace{43mm} = \left(-10 - 54\right) \\
	\phantom{6} \hspace{43mm} = -64 \\
	$
	bulunur.
\end{cozum}

\begin{ornek}
	\quad \vspace{3mm} \\
	$
	A=\left[
	\begin{array}{lcr}
		-4           & -2           & \phantom{-}3 \\
		\phantom{-}2& \phantom{-}1  &           -1  \\
		\phantom{-}6 & \phantom{-}3  & \phantom{-}2
	\end{array}\right]_{3\ x\ 3}
	$
	matrisinin determinantını hesaplayınız.
\end{ornek}

\begin{cozum}
	\quad \\
	$
	det[A]=
	\begin{aligned}
		\begin{asy}
			size(3cm,0);
			pair z1=(1,1);
			pair z2=(1,0);
			pair z3=(1,-1);
			pair z4=(2,1);
			pair z5=(2.0);
			pair z6=(2,-1);
			pair z7=(3,1);
			pair z8=(3,0);
			pair z9=(3,-1);
			pair z10=(1,-2);
			pair z11=(1,-3);
			pair z12=(2,-2);
			pair z13=(2,-3);
			pair z14=(3,-2);
			pair z15=(3,-3);
			label("$-4$",z1);
			label("$2$",z2);
			label("$6$",z3);
			label("$-2$",z4);
			label("$1$",z5);
			label("$3$",z6);
			label("$3$",z7);
			label("$-1$",z8);
			label("$2$",z9);
			label("$-4$",z10);
			label("$2$",z11);
			label("$-2$",z12);
			label("$1$",z13);
			label("$3$",z14);
			label("$-1$",z15);
			draw((0.8,-1.5)--(3.2,-1.5));
			draw((0.5,1.3)--(0.5,-1.3));
			draw((3.5,1.3)--(3.5,-1.3));
			draw(z9--(0,1.8),Arrow,PenMargins);
			draw(z14--(0,0.8),Arrow,PenMargins);
			draw(z15--(0,-0.2),Arrow,PenMargins);
			draw(z7--(0,-1.7),red,Arrow,PenMargins);
			draw(z8--(0,-2.7),red,Arrow,PenMargins);
			draw(z9--(0,-3.7),red,Arrow,PenMargins);
			label("$+$",(-0.5,1.8));
			label("$+$",(-0.5,0.8));
			label("$+$",(-0.5,-0.2));
			label("$-$",(-0.5,-1.7));
			label("$-$",(-0.5,-2.7));
			label("$-$",(-0.5,-3.7));
		\end{asy}
	\end{aligned}
	=\left[(-4) \cdot 1 \cdot 2 + 2 \cdot 3 \cdot 3 + 6 \cdot (-2) \cdot (-1) \right]
	-\left[6 \cdot 1 \cdot 3 + (-4) \cdot 3 \cdot (-1) + 2 \cdot (-2) \cdot 2 \right] \\
	\phantom{6} \hspace{43mm} = \left(-8+18+12\right)-\left(18+12-8\right) \\
	\phantom{6} \hspace{43mm} = \left(22 - 22\right) \\
	\phantom{6} \hspace{43mm} = 0 \\
	$
	bulunur.
\end{cozum}
\subsection{nxn Tipindeki Karesel Matrislerin Determinantı}
\begin{tanim}
	\quad \\
	$A=\left[a_{ij}\right]_{nxn}$ nxn tipinde bir karesel matris olmak üzere,$A$'nın herhangi bir $a_{ij}$
	elemanının bulunduğu satır ve sütun silindiğinde kalan matrisin determinantına $a_{ij}$ elemanının
	Minörü denir ve $M_{ij}$ şeklinde gösterilir.
\end{tanim}

\begin{tanim}
	$A=\left[a_{ij}\right]_{nxn}$ nxn tipinde bir karesel matris ve $a_{ij}$ elemanının minörü $M_{ij}$ olsun.
	$$
	A_{ij}=\left(-1\right)^{i+j} \cdot M_{ij}
	$$
	sayısına $a_{ij}$ elemanının Kofaktörü denir.
\end{tanim}

\begin{ornek}
	\quad \vspace{3mm} \\
	$
	A=\left[
	\begin{array}{lcr}
		-3           & -1           & \phantom{-}3\  \\
			     &	            & \              \\
		\phantom{-}1 & \phantom{-}2 & -1\            \\
		       	     &              & \              \\
		\phantom{-}2 & \phantom{-}3 & -2\ 
	\end{array}\right]_{3\ x\ 3}
	$
	matrisinin determinantını hesaplayınız.
\end{ornek}

\begin{cozum}
	\quad \\
	$
	det[A]=
	\begin{aligned}
		\begin{asy}
			size(3cm,0);
			pair z1=(1,1);
			pair z2=(1,0);
			pair z3=(1,-1);
			pair z4=(2,1);
			pair z5=(2.0);
			pair z6=(2,-1);
			pair z7=(3,1);
			pair z8=(3,0);
			pair z9=(3,-1);
			label("$-3$",z1);
			label("$1$",z2);
			label("$2$",z3);
			label("$-1$",z4);
			label("$2$",z5);
			label("$3$",z6);
			label("$3$",z7);
			label("$-1$",z8);
			label("$-2$",z9);
			draw((0.8,-1.5)--(3.2,-1.5));
			draw((0.5,1.3)--(0.5,-1.3));
			draw((3.5,1.3)--(3.5,-1.3));
			draw(z2--z8,dotted);
			draw(z4--z6,dotted);
		\end{asy}
	\end{aligned}
	=\left[(-4) \cdot 1 \cdot 2 + 2 \cdot 3 \cdot 3 + 6 \cdot (-2) \cdot (-1) \right]
	-\left[6 \cdot 1 \cdot 3 + (-4) \cdot 3 \cdot (-1) + 2 \cdot (-2) \cdot 2 \right] \\
	\phantom{6} \hspace{43mm} = \left(-8+18+12\right)-\left(18+12-8\right) \\
	\phantom{6} \hspace{43mm} = \left(22 - 22\right) \\
	\phantom{6} \hspace{43mm} = 0 \\
	$
	bulunur.
\end{cozum}
\begin{center}
	\begin{asy}
import math;
import graph;
size(6cm,0);
real f(real x) {return sin(x);}
pair F(real x) {return (x,f(x));}
xaxis("$x$",red,Arrow);
yaxis("$y$",red,Arrow);
draw(graph(f,-3.0,0.7,operator ..));
draw(graph(f,1.3,3.0,operator ..));
label("$\frac{x^3-3x}{x-1}$",F(3.0),NW);
	\end{asy}
\end{center}
\end{document}
