\documentclass[twocolumn]{article}
\usepackage[utf8x]{inputenc}
\usepackage{amsmath}
\usepackage{mathptmx}
\pagestyle{empty}
\voffset=-10.0mm
\hoffset=-0.5in
\footskip=-40mm
\topmargin=5mm
\headheight=0.mm
\headsep=0.mm	
\oddsidemargin=-5mm		
\evensidemargin=-5mm		
\textheight=250mm
\textwidth=200mm
\setlength{\tabcolsep}{0mm}
\DeclareMathSizes{12}{12}{8}{7}
\begin{document}
\noindent \underline{\textbf{Kerameti kendinden menkul}}:\textsf{ Sahip olduğu nitelikleri kendisi söyleyen. }\textit{"Kerameti kendinden menkul şeyhler gibi bu armağanlar onların eksik olan kabiliyetlerinin bir çeşit icazeti oluyor."} \\
\noindent \underline{\textbf{Satır aralarında}} \\
\noindent \underline{\textbf{Tevessül etmek}}:\textsf{ Başlamak,girişmek. }\textit{"Kendi zaafını mağduriyet gibi göstermeye tevessül etmesinler"} \\
\noindent \underline{\textbf{Dirayet}}:\textsf{ Zeka. }\textit{"Bu ülkeye eli sopalı, bileğine kuvvetli, dirayetli biri lazım."} \\
\noindent \underline{\textbf{Meleke}}:\textsf{ Tekrarlama sonucu kazanılan yatkınlık,alışkanlık. }\textit{"Tahmin yapmaya yapmaya ya bu melekem büsbütün körleşirse"} \\
\noindent \underline{\textbf{Ehven-i Şer}}:\textsf{ Kötünün iyisi. }\textit{} \\
\noindent \underline{\textbf{Acuze}}:\textsf{ Huysuz,yaşlı kadın. }\textit{"Korkunç bir acuze onu kucaklamaya çalışıyordu."} \\
\noindent \underline{\textbf{İfrtit Olmak(kesilmek)}}:\textsf{ Çok öfkelenmek;çok kızmak. }\textit{"Arzuma karşı konulunca ifrit kesildiğimi pek iyi bildiğinden ses çıkarmadı"} \\
\noindent \underline{\textbf{Maraza Çıkarma}}:\textsf{ Kavgaya yol açmak,kavga çıkarmak,anlaşmazlığa yol açacak işler yapmak. }\textit{"Dazlak kafası ve seyrek sakallarıyla gezinen, ikide bir maraza çıkaran genç lumpenlere çok mu bayılıyordu da"} \\
\noindent \underline{\textbf{Mezkur}}:\textsf{ Adı geçen, anılan, sözü geçen, zikredilen. }\textit{"Mezkûr evi de kiraya vermiştim."} \\
\noindent \underline{\textbf{Madrabaz}}:\textsf{ Hile yapan, hileci. }\textit{"Bunlar kusurlu, adi camlardır, köy evi camları! Madrabazlar böylelerini köylere satarlar."} \\
\noindent \underline{\textbf{İstiskal Etmek}}:\textsf{ Hoşnutsuzluğunu belli ederek soğuk davranmak }\textit{"Şahin Efendi, bu saygısız misafiri artık açıktan açığa istiskal ediyordu."} \\
\noindent \underline{\textbf{Kaltaban}}:\textsf{ Namussuz; Şarlatan, yalancı, hileci.}\textit{} \\
\noindent \underline{\textbf{Şiar}}:\textsf{ Ülkü,düstur. }\textit{} \\
\noindent \underline{\textbf{Tefrika}}:\textsf{ Birbirine kötülük etmeye kadar varan sürekli anlaşmazlık, ikilik. }\textit{"Türkler arasına tefrika ve nifak koymak için de hürriyet vermişti."} \\
\noindent \underline{\textbf{Mütevekkil}}:\textsf{ Her işini Tanrı'ya veya oluruna bırakmış, kadere boyun eğmiş. }\textit{"Komşularının bu mütevekkil hâline pek şaştı."} \\
\noindent \underline{\textbf{Gudubet}}:\textsf{ Yüzüne bakılmayacak kadar sevimsiz ve çirkin. }\textit{} \\
\noindent \underline{\textbf{Düçar Olmak}}:\textsf{ Uğramak, yakalanmak, tutulmak. }\textit{"Nitekim musibet zamanlarında iyi, refah zamanlarında kötü hasletlere düçar olmamız, biraz da bu vasıflarımızla alâkalıdır."} \\
\noindent \underline{\textbf{Sarih}}:\textsf{ Açık, kolay anlaşılır, belli belirgin, belgin. }\textit{"O zaman Müfit'i sarih bir şüphe yakaladı."} \\
\noindent \underline{\textbf{Temayüz Etmek}}:\textsf{ Sivrilmek, seçkinleşmek. }\textit{"Nefis sanatlarda kolaylıkla temayüz edebilirsiniz."} \\
\noindent \underline{\textbf{Müstear}}:\textsf{ Takma. }\textit{"Müstear adların hakikilerini saymaya başladı."} \\
\noindent \underline{\textbf{Metrukat}}:\textsf{ Ölen birinin bıraktığı şeyler. }\textit{} \\
\noindent \underline{\textbf{Sekter}}:\textsf{ Katı, hoşgörüsüz (düşünce, tutum). }\textit{} \\
\noindent \underline{\textbf{Menfur}}:\textsf{ Nefret edilen, iğrenç, tiksindirici. }\textit{"Menfur saldırı."} \\
\noindent \underline{\textbf{Meşum}}:\textsf{ Uğursuz }\textit{"İlk çocuğunu doğuran genç bir kadına meşum şeyler söylememeliydim, sustum."} \\
\noindent \underline{\textbf{İçtihat}}:\textsf{ Yasada veya örf ve âdet hukukunda uygulanacak kuralın açıkça ve tereddütsüz olarak bulunmadığı konularda, yargıcın veya hukukçunun düşüncelerinden doğan sonuç. }\textit{} \\
\noindent \underline{\textbf{Muvazaa}}:\textsf{ Danışıklık. }\textit{"Muvazaalı işlem hukuka aykırıdır çünkü irade neyse beyanda o olmalıdır"} \\
\noindent \underline{\textbf{İfrata Kaçmak}}:\textsf{ Çok ileri gitmek, aşırı davranmak. }\textit{"Onlar ifrata kaçmışların bir numunesi olduklarını bilmezden gelirler"} \\
\noindent \underline{\textbf{Tefrit}}:\textsf{ Herhangi bir konuda geride kalma, yeterli ölçüde olmama durumu. }\textit{ } \\
\noindent \underline{\textbf{Fevç Fevç}}:\textsf{ Akın akın. }\textit{"İnsanlar Allah'in dinine fevc fevc girdiler, ondan fevc fevc çikacaklar. " } \\
\noindent \underline{\textbf{Sabık}}:\textsf{ Geçen, önceki, eski. }\textit{"Yorucu çalışmalar sonunda sabık bakanların ne derece hüner sahibi olduklarını tespit etmiştir."} \\
\noindent \underline{\textbf{Ukde}}:\textsf{ İçe dert olan şey. }\textit{"Yazarlık hayatım boyunca içimde bir ukde olarak kalmıştı."} \\
\noindent \underline{\textbf{Kıl Kuyruk}}:\textsf{ 1.Zayıf, çelimsiz. 2.Züğürt, kılıksız. 3.Niteliksiz. }\textit{"Hiçbir özelliği olmayan, kendi hâlinde, gösterişsiz, kıl kuyruk bir kedi idi." } \\
\noindent \underline{\textbf{İkrah Etme }}:\textsf{ Tiksinme,iğrenme. }\textit{ } \\
\noindent \underline{\textbf{De-Fakto}}:\textsf{ Etik olma zorunluluğu da, yasal olma zorunluğu da olmamasına rağmen uygulama zorunluluğu varmış gibi olan ancak uygulayan rolündeki taraf için kesinlikle uygulanma sebebi bir temele oturtulmuş olan bişeydir. }\textit{"Almanya yenilince biz de yenik sayıldık en büyük de facto örneği olabilir."} \\
\noindent \underline{\textbf{Sıdkı Sıyrılmak}}:\textsf{ Birine karşı duyulan güven ve inancı yitirmek. }\textit{"Adına en soylu dileklerde bulunduğumuz bu bağırgan, kaba ve düşüncesiz insan yığınından, o dakikada sıdkım sıyrılmaya yetti."} \\
\noindent \underline{\textbf{Avane}}:\textsf{ Kötü işlerde birine yardım eden kimse,yardakçı. }\textit{"Tıpkı Saddam Hüseyin gibi bunlar da küresel katil olan Amerika ve avaneleri olan diğer emperyalistleri sırtlarında taşıyan koşu atlarıdır."} \\
\noindent \underline{\textbf{Akredite }}:\textsf{ }\textit{ } \\
\noindent \underline{\textbf{Duhul}}:\textsf{ Girme,giriş. }\textit{"Gakkoş Mehmet’in saflarına duhul etti." } \\
\noindent \underline{\textbf{Kervan Yolda Dizilir}}:\textsf{ }\textit{ } \\
\noindent \underline{\textbf{Nehir Ortasında At Değiştirilmez}}:\textsf{ }\textit{ } \\
\noindent \underline{\textbf{Resen}}:\textsf{ Kendi başına, kendiliğinden. }\textit{"Emniyet resen harekete geçmeli."} \\
\noindent \underline{\textbf{Zinhar}}:\textsf{ Asla }\textit{"Fakat adını zinhar benden öğrenemeyeceksin." } \\
\noindent \underline{\textbf{Kerhen}}:\textsf{ İstemeyerek, istemeye istemeye, gönülsüz olarak. Emrivakiiyle yerine getirilen }\textit{ } \\
\noindent \underline{\textbf{Beis Görmemek}}:\textsf{ Sakınca, zarar görmemek. }\textit{"Seyyit Ali, Yani'ye planlarını üstünkörü anlatmakta beis görmedi." } \\
\noindent \underline{\textbf{Tapon}}:\textsf{ Niteliği düşük, eski, elde kalmış. }\textit{"İşportanın içinde hayatın en yırtık, en tapon matahları satılıyordu." } \\
\noindent \underline{\textbf{Matah}}:\textsf{ İnsan, mal, eşya vb. için küçümseme yollu bir söz. }\textit{"Kadının çantası da matah bir şey değil zaten."} \\
\noindent \underline{\textbf{Velev ki}}:\textsf{ İster, isterse, olsa da, kaldı ki, hatta. }\textit{ } \\
\noindent \underline{\textbf{Evla}}:\textsf{ Daha iyi, yeğ }\textit{"Bir şeyi bilmek, onun cahili olmaktan evladır, diyen bir hadis vardır."} \\
\noindent \underline{\textbf{Bu Başaktan Bu Kadar Tane Çıkar}}:\textsf{ }\textit{ } \\
\noindent \underline{\textbf{Çam Devirmek}}:\textsf{ Karşısındakine dokunacak veya kötü bir sonuç doğuracak söz söylemek. }\textit{"Bu hoppa oğlan, karısına ne diller dökecek, ne potlar kıracak, ne çamlar devirecekti."} \\
\noindent \underline{\textbf{Hırpani}}:\textsf{ Perişan kılıklı, derbeder. }\textit{"Yanımıza elleri saygı ile göbeğinin altına bağlı, hırpani bir delikanlı yanaştı."} \\
\noindent \underline{\textbf{İbadullah}}:\textsf{ Ucuz veya bol olan şey. }\textit{"Pazarda balık ibadullah"} \\
\noindent \underline{\textbf{Sulta}}:\textsf{ Yaptırma, yasak etme, emretme, itaat ettirme hakkı veya gücü, yetke, velayet }\textit{"Ak Parti, ilk kurulduğunda tüzüğüne demokrasi kavramını, taban demokrasisini ve lider sultası yerine ortak aklı özümseyen bir parti olacağını yazmıştı."} \\
\noindent \underline{\textbf{Lime Lime}}:\textsf{ Parça parça. }\textit{ } \\
\noindent \underline{\textbf{Daniska}}:\textsf{ Ala. }\textit{"Bu sade dile değil, karşısındakine de saygısızlığın daniskası."} \\
\noindent \underline{\textbf{Netameli}}:\textsf{ Gizli bir tehlikesi olduğu sanılan, tekin olmayan. }\textit{"Artık yürüyelim, bir an önce çıkalım bu netameli yerden."} \\
\noindent \underline{\textbf{Mutat}}:\textsf{ Alışılmış, alışılan. }\textit{"Cumhurbaşkanı, Başbakan Ecevit'i mutat görüşme günü olan dünkü perşembe günü kabul etmeyeceğini bildiriyor."} \\
\noindent \underline{\textbf{Aynıyla Vaki}}:\textsf{ Var böyle bir şey.Tıpkı aynısı var. }\textit{"Elin oğlunun uyarısı, bizimkilerin bir kulağından giriyor, ötekinden çıkıyor. Adamın dediği şimdi aynıyla vaki."} \\
\noindent \underline{\textbf{Beynamaz}}:\textsf{ Hiç namaz kılmayan kimse.Sorumluluk almayan. }\textit{ } \\
\noindent \underline{\textbf{Kumkuma}}:\textsf{ Kötü, olumsuz bir özelliği kendinde fazlasıyla toplayan kimse, olay, olgu veya yer. }\textit{"Burnundan kıl aldırmayacak kadar kompleks kumkuması bir adamdı."} \\
\noindent \underline{\textbf{Tahakküm Etmek}}:\textsf{ Baskı yapmak, zorbalık etmek, hükmetmek. }\textit{"Hürriyet odur ki, adil konular dışında bir kimse kimseye tahakküm etmesin."} \\
\noindent \underline{\textbf{Zürriyet}}:\textsf{ Döl, soy sop, sulp }\textit{"kulakları sağır eden, adamı zürriyetinden kesen bir hiddet çığlığı"} \\
\noindent \underline{\textbf{Afaki}}:\textsf{ Belli bir konu üzerine olmayan, dereden tepeden (konuşma) }\textit{"Siz afaki konusurken, arguman cakiyoruz"} \\
\noindent \underline{\textbf{Cadı Avı}}:\textsf{ }\textit{ " Yeni mesihçiliğin yayılması karşısında Kilise, bu kehanetleri cadıların yaydığını söyleyip, korkunç bir cadı avı başlatmıştı." } \\
\noindent \underline{\textbf{Parsa}}:\textsf{ Bir izleyici topluluğu önünde yapılan gösteriden sonra toplanan para. }\textit{"Tüm düzen partileri, bu milliyetçi-şoven gerici saldırıları özel bir tarzda körükleyerek parsadan pay kapmak için birbirleriyle yarışıyorlar."} \\
\noindent \underline{\textbf{Davulu Biz Çaldık Parsayı Başkası Topladı}}:\textsf{ }\textit{ } \\
\noindent \underline{\textbf{Minval}}:\textsf{ Biçim, yol, tarz. }\textit{Eusobie Leal Spengler, geçtiğimiz Salı günü Çekül Vakfı’nda davetlilerle buluştu. Spengler’i takdim eden Prof. Metin Sözen, kültür öncelikli kalkınmanın yollarını aramız gerektiğini Çekül Vakfı’nın çalışmalarının da bu minvalde yürüdüğünü söyledi.} \\
\noindent \underline{\textbf{Kisvesi altında}}:\textsf{ Herhangi bir nitelikte veya biçimde anlamında kullanılan bir söz. }\textit{"Ayrımcılık ve eşitsizlik din kisvesi altında kutsanabilir mi?" } \\
\noindent \underline{\textbf{Safsata}}:\textsf{ Boş, temelsiz, asılsız söz. }\textit{"Türk Cumhuriyeti, varlığını, istiklalini safsatalarla tehlikeye maruz bırakamaz." } \\
\noindent \underline{\textbf{Nezahet}}:\textsf{ Temizlik, ahlak temizliği. }\textit{"Şimdi ne nezahet kaldı ne nezaket." } \\
\noindent \underline{\textbf{Done}}:\textsf{ Veri. }\textit{"Kritiği bıraktım, kısıtlı donelerle, tespitlerle en zor tür olan eleştiri bile yazmaya kalkıyorlar."} \\
\noindent \underline{\textbf{Kurt Dumanlı Havayı Sever}}:\textsf{ }\textit{ } \\
\noindent \underline{\textbf{Çıngar Çıkarmak}}:\textsf{ Gürültü, kavga çıkarmak. }\textit{"Yarışmada çıngar çıkartan, Armağan Çağlayan'ın.." } \\
\noindent \underline{\textbf{Mavra}}:\textsf{ Gevezelik,palavra. }\textit{ } \\
\noindent \underline{\textbf{Mushaf}}:\textsf{ Kur-anı Kerim. }\textit{ } \\
\noindent \underline{\textbf{Müphem}}:\textsf{ Belirsiz,örtülü. }\textit{""Müphem konuşuyor."} \\
\noindent \underline{\textbf{Tezvirat}}:\textsf{ Yalan dolan şeyler, kovculuklar. }\textit{"Gazeteler ve kanal, şimdiye kadar eşi benzeri görülmemiş bir tezvirat kampanyası eşliğinde iki günlük hükümetin üzerine gidiyorlar."} \\
\noindent \underline{\textbf{Canhıraş}}:\textsf{ Yürek paralayan, kulak tırmalayan, acı, tüyler ürpertici. }\textit{Canhıraş bir feryat koparır koparmaz, ipek gömlekle odaya kendimi atmışım."} \\
\noindent \underline{\textbf{Alarga}}:\textsf{ Uzaktan, açıktan. }\textit{"Arkadaşlarımdan mümkün olduğu kadar alarga yürüyor, kendimi pencerelerin, kapıların ışık sahası dışına çıkarmaya uğraşıyordum."} \\
\noindent \underline{\textbf{Patronaj}}:\textsf{ Cezaevinden serbest bırakılan suçlunun toplum yaşantısına yeniden uyabilmesini sağlamak amacıyla yapılan yardım çalışması. }\textit{"Kapitalist sistemin doğurduğu patronaj ilişkileri ve büyük rant paylaşımı kentlerde gelişmeyle birlikte çelişkilerin büyümesine de yol açmış, küçük burjuvazi yani orta sınıfları bile emekçiler ve alt gelir gruplarıyla birlikte çağdaş yaşamın dışına iteklemiştir."} \\
\noindent \underline{\textbf{İpe Un Sermek}}:\textsf{ Geçersiz birtakım nedenler ileri sürerek istenilen işi yapmaktan kaçınmak. }\textit{ } \\
\noindent \underline{\textbf{Defaten}}:\textsf{ Birden, aniden, bir çırpıda bir kerede. }\textit{"Sönük bakan gözleri defaten parladı."} \\
\noindent \underline{\textbf{Defaatle}}:\textsf{ Çok kez,çok kere. }\textit{"Daha öncede defaatle söylediğim gibi satranç aynı zamanda dinamik bir spor olmanın yanı sıra bir bilimdir."} \\
\noindent \underline{\textbf{Nedamet}}:\textsf{ Pişmanlık. }\textit{""Nedametle, kendimden ve etrafımdakilerden tiksinti içinde inzivama dönerim."} \\
\noindent \underline{\textbf{Şifahen}}:\textsf{ Ağızdan, sözle söyleyerek. }\textit{"Şifahen emretti."} \\
\noindent \underline{\textbf{Şirazeden Çımak}}:\textsf{ Düzenini kaybetmek, çığırından çıkmak. }\textit{"Eskilerin dediği gibi bu iş şirazeden çıktı."} \\
\noindent \underline{\textbf{Kelalaka}}:\textsf{ -hiç ilgisi yok, ne ilgisi var- anlamlarında kınama yollu kullanılır. }\textit{ } \\
\noindent \underline{\textbf{Tenevvür Etmek}}:\textsf{ Aydınlanmak. }\textit{"Voltaire ve ansiklopedistler Fransa'da aklın hâkimiyetini kabul eden bir tenevvür devri yaratmışlardı."} \\
\noindent \underline{\textbf{Sermayeyi Kediye Yüklemek}}:\textsf{ Boşa yatırım yapmayı anlatan deyim. }\textit{"Ama doğrusu, kediye sermaye yatıracak kadar zengin olmadığımdan bu mevcudiyetin daha ne kadar müddet devam edeceği konusunda bahse girmek rizikosunu göze alamam."} \\
\noindent \underline{\textbf{Vesvese}}:\textsf{ Kuruntu. }\textit{"Etrafı su olduğu için acaba kökünü bırakıp yüzüverir mi, diye içime bir vesvese girer."} \\
\noindent \underline{\textbf{Maiyet}}:\textsf{ Üst görevlinin yanında bulunan kimseler, alt kademedekiler. }\textit{"Amir, maiyetindeki çalışanlara hakkaniyet ve eşitlik içinde davranır."} \\
\noindent \underline{\textbf{Kulağına Kar Suyu Kaçarız}}:\textsf{ Şüphelenmek. }\textit{ } \\
\noindent \underline{\textbf{Kavil}}:\textsf{ Söz. }\textit{"Babamın kavline göre, bu adam bütün Manisa halkını iki büyük afetten kurtarmış."} \\
\noindent \underline{\textbf{ }}:\textsf{ }\textit{ } \\
\noindent \underline{\textbf{ }}:\textsf{ }\textit{ } \\
\noindent \underline{\textbf{ }}:\textsf{ }\textit{ } \\
\noindent \underline{\textbf{ }}:\textsf{ }\textit{ } \\
\noindent \underline{\textbf{ }}:\textsf{ }\textit{ } \\
\noindent \underline{\textbf{ }}:\textsf{ }\textit{ } \\
\noindent \underline{\textbf{ }}:\textsf{ }\textit{ } \\
\noindent \underline{\textbf{ }}:\textsf{ }\textit{ } \\
\noindent \underline{\textbf{ }}:\textsf{ }\textit{ } \\
\noindent \underline{\textbf{ }}:\textsf{ }\textit{ } \\
\noindent \underline{\textbf{ }}:\textsf{ }\textit{ } \\
\noindent \underline{\textbf{ }}:\textsf{ }\textit{ } \\
\noindent \underline{\textbf{ }}:\textsf{ }\textit{ } \\
\noindent \underline{\textbf{ }}:\textsf{ }\textit{ } \\
\noindent \underline{\textbf{ }}:\textsf{ }\textit{ } \\
\noindent \underline{\textbf{ }}:\textsf{ }\textit{ } \\
\noindent \underline{\textbf{ }}:\textsf{ }\textit{ } \\
\noindent \underline{\textbf{ }}:\textsf{ }\textit{ } \\
\noindent \underline{\textbf{ }}:\textsf{ }\textit{ } \\
\noindent \underline{\textbf{ }}:\textsf{ }\textit{ } \\
\noindent \underline{\textbf{ }}:\textsf{ }\textit{ } \\
\noindent \underline{\textbf{ }}:\textsf{ }\textit{ } \\
\noindent \underline{\textbf{ }}:\textsf{ }\textit{ } \\









\end{document}
